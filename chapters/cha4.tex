% chapters/cha4.tex

\chapter{行列式}
\label{chap:4}

\begin{center}
  随着你离开的越来越久,我也慢慢变成了大人……也许你已经忘了吧……

{\footnotesize 君がいない日々を超えて,あたしも大人になっていく……こうやって全て忘れていくのかな……}  

\end{center}
\rightline{{\small ——いきものがかり《SAKURA》}}
\vspace{-5pt}
\begin{center}
    \pgfornament[width=0.36\linewidth,color=lsp]{88}
  \end{center}
\section{对二阶行列式的讨论}
\label{sec:41}



\begin{definition}{二阶行列式} \label{def:SecondOrderDeterminant}
  设有二阶方阵\[
\bm{A} =
\begin{pmatrix}
  \bm{\alpha}_{1} & \bm{\alpha}_{2} \\
\end{pmatrix}
=
\begin{pmatrix}
  a_{11} & a_{12} \\
  a_{21} & a_{22} \\
\end{pmatrix}
\]
若定义在$\bm{A}$或向量$\bm{\alpha}_{1}$,$\bm{\alpha}_{2}$上的函数\(\delta ( \bm{\alpha}_{1},\bm{\alpha}_{2} \) 满足:
  
\end{definition}

% 命题 4.1.3
\begin{proposition}{二阶行列式的计算公式}
  设有二阶矩阵\(\bm{A} = (a_{ij})_{2 \times 2}\),则$\bm{A}$的行列式满足\[
\mathrm{det} \bm{A} = a_{11}a_{22} - a_{21}a_{12}
    \]
  \end{proposition}

  % 命题4.1.4
  \begin{proposition}
    若\(\mathrm{rank} \bm{A} < 2 \),则\(\mathrm{det} \bm{A} = 0 \)
  \end{proposition}

  % 命题4.1.5
  \begin{proposition}{二阶行列式的变换}
    已知$\bm{P}_{1 \leftrightarrow 2},\bm{P}_{ii;k}(k \neq 0),\bm{P}_{ij;k}(i \neq j)$分别表示2阶对换矩阵,倍乘矩阵以及倍加矩阵,则对任意的二阶方阵\(\bm{A} = (\bm{\alpha}_{1} ,\bm{\alpha}_{2} = (a_{ij})_{2 \times 2} \)与任意的$i,j=1,2$则有:
    \begin{enumerate}
    \item \( \)
    \end{enumerate}
  \end{proposition}



%%% Local Variables:
%%% mode: LaTeX
%%% TeX-master: "../main.tex"
%%% coding: utf-8
%%% End:
