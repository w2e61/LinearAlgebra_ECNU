% chapters/cha2.tex

\chapterimage{Hyouka1.png}
\chapter{矩阵}
\label{chap:2}

\section{映射的运算}
\label{sec:21}
need to be completed

\section{矩阵的运算}
\label{sec:22}
need to be completed

\subsection{加法和数乘}
\label{sec:221}

我们需要找到一个矩阵是另两个矩阵$\bm{A},\bm{B}$的和,也就是说需要找到符合条件的矩阵$(\bm{A}+\bm{B})$,使\[
\forall \bm{x} , (\bm{A}+\bm{B})\bm{x} = \bm{Ax} + \bm{Bx} \]
这个表达非常的含糊其辞:什么叫$\forall x$?

need to be completed

\subsection{乘法(复合)}
\label{sec:222}

现在我们来讨论由矩阵引导的线性映射的复合,即需要对$\forall \bm{x}$求$\bm{A}(\bm{Bx})$,为此,需要保证$\bm{Bx}$的陪域在$\bm{A}$的定义域之内,因而两者需要有相同的维数,亦即说:
\begin{center}
  \textbf{外映射的列数必须等于内映射的行数},这也就是中间向量的维数。
\end{center}\[
  \begin{array}[H]{*{5}{c} l}
    \bm{AB}:& \mathbb{R}^{n} & \xrightarrow{\bm{B}} & \mathbb{R}^{s} & \xrightarrow{\bm{A}} & \mathbb{R}^{m}\\
            & \bm{x}         & \mapsto           & \bm{y} = \bm{Bx} & \mapsto & \bm{z} = \bm{Ay} = \bm{A}(\bm{Bx}) = (\bm{AB} )\bm{x} \\
  \end{array}
\]
\begin{definition}{矩阵的乘法}
  \label{def:MultiplicationOfMatrix}
  设有矩阵$\bm{A} = (a_{ik})_{m \times s}$和\( \bm{B} = (b_{kj})_{s \times n} \),则矩阵$\bm{A}$和$\bm{B}$的复合,或被称为\textbf{矩阵的乘法}为:
  \begin{equation}
    \label{eq:MultiplicationOfMatrix}
    \bm{AB} = \left( \sum^{s}_{k=1} \left( a_{ik}b_{kj} \right) \right)_{m\times n}
  \end{equation}
\end{definition}
\begin{remark}
  事实上,上述定义~\ref{def:MultiplicationOfMatrix}是可以由矩阵左乘向量的定义~\ref{def:LeftMultiplicationOfMatrix}推出的,如下所示:
  \begin{align*}
    \bm{A}(\bm{Bx}) &= (a_{ik})_{m \times s} \left( \sum_{j=1}^{n}\left( k_{kj}x_{j} \right) \right)_{s \times 1}\\
                    &= \left( \sum_{k=1}^{s} a_{ik} \sum_{j=1} \right)_{m \times 1} \\
                    &= \left( \sum_{j=1}^{n} \left( \sum_{k=1}^{s}\left(  a_{ik}b_{kj} \right) \right) x_{j} \right)_{m \times 1}\\
                    &= \left( \sum^{s}_{k=1} \left( a_{ik}b_{kj} \right) \right)_{m\times n} \bm{x}
  \end{align*}
\end{remark}
\begin{note}
  通过上述矩阵乘法的定义(由数字给出),可以直接发现以下两点:
  \begin{enumerate}
  \item   \textbf{外映射的列数必须等于内映射的行数},这是矩阵乘法有意义的充要条件
  \item 在$\bm{AB}$有意义的前提下,乘积矩阵$\bm{AB}$的行数和列数分别等于$\bm{A}$的行数和$\bm{B}$的列数,且其$(i,j)$元素等于$\bm{A}$的第i行与$\bm{B}$的第j列对应元素之和。\label{item:MultiplicationOfMatrix1}

  \end{enumerate}
\end{note}

接下来,我们来考察矩阵乘法的性质。

%例2.2.6
\begin{example}
  need to be completed 例2.2.6
\end{example}
\begin{remark}
  这个例子告诉我们,矩阵乘法一般不满足\textbf{乘法交换律}。即$\bm{AB} = \bm{BA}$不一定成立。
\end{remark}

% 例2.2.7
\begin{example}
  need to be completed 例2.2.7
\end{example}
\begin{remark}
  这个例子告诉我们,矩阵乘法一般不满足\textbf{乘法消去律}。即\textbf{不能}通过$\bm{AC} = \bm{BC}$和$\bm{C} \neq \bm{0}$推出$\bm{A} = \bm{B}$。后文将会知道,这其实需要补充条件:$\bm{C}$是可逆的。
\end{remark}

% 例2.2.8
\begin{example}
  need to be completed 例2.2.8
\end{example}
\begin{remark}
  这个例子告诉我们,已知$\bm{AB} = \bm{0}$,也\textbf{不能}得到$\bm{A} = \bm{0}$或$\bm{B} = bm{0}$。也可以认为此例是上例的特殊情况,即不能通过消去一个矩阵而得到另一个矩阵必为0的矩阵。同样的,如果我们补充了其中一个矩阵是可逆的,就可以得到另一个矩阵是零矩阵的结论。
\end{remark}

接下来,我们从向量的角度看待矩阵的乘法。在第~\ref{sec:1321}页中我们指出,可以将矩阵看成由多个行(列)向量排成的阵列。接下来用向量的方法表述矩阵的乘法:
\begin{definition}{矩阵的乘法(向量)}
  \label{def:MultiplicationOfMatrixByVectors}
  \begin{enumerate}
  \item 前文~\ref{item:MultiplicationOfMatrix1}即是说,乘积矩阵$\bm{AB}$的$(i,j)$元素等于$\bm{A}$的第i行构成的向量点乘$\bm{B}$的第j列构成的向量。
  \item $\bm{AB}$的第i列,是$\bm{A}$的每一列的线性组合,以$\bm{B}$的第i列作为组合系数。
  \item $\bm{AB}$的第j行,是$\bm{B}$的每一行的线性组合,以$\bm{A}$的第j行作为组合系数。
  \item 乘积矩阵是$\bm{A}$的每一列和$\bm{B}$的每一行的乘积之和。(矩阵的秩一分解) %练习3.6.8
  \end{enumerate}
\end{definition}

\begin{note}
  第二、第三种表述可以认为是矩阵的分块运算(将在~\ref{sec:24}中提及);
  
  第一种表述可以认为是第二和第三种表述的展开表述;
  
  第四种表述和第一种表述是完全相对的。
\end{note}

%命题2.2.10
\begin{proposition}{矩阵乘法的性质}

  \begin{enumerate}
  \item 零矩阵
  \item 单位矩阵
  \item 乘法结合律
  \item 对加法的分配律
  \item 对数乘的交换律
  \end{enumerate}
\end{proposition}


上文我们提到,$\bm{AB} = \bm{BA}$不一定成立,但我们仍然可以给出以下一个定义和三个命题:

% 定义2.2.11
\begin{definition}{矩阵的相乘可交换}
  
\end{definition}

% 命题2.2.12
\begin{proposition}{可交换的充分条件}
  
\end{proposition}

% 命题2.2.13
\begin{proposition}{可交换的必要条件}
  
\end{proposition}

% 命题2.2.14
\begin{proposition}
  若$\bm{D}_{1}$和$\bm{D}_{2}$为同阶对角阵,则必有$\bm{D}_{1} \bm{D}_{2} = \bm{D}_{2} \bm{D}_{1}$
\end{proposition}

\subsection{方幂}
\label{sec:223}

\begin{definition}{矩阵的方幂}
  \label{def:PowerOfMatrix}
  need to be completed soon
\end{definition}

对于特殊矩阵,有一些便捷的计算方幂的方法,如下所示:

% 例2.2.15
\begin{example}
  设对角矩阵\(\bm{D} = \mathrm{diag}(d_{1},d_{2},\cdots,d_{n} ,l \in \mathbb{N}^{+}\),计算$\bm{D}^{l}$
\end{example}
\begin{solution}
  通过归纳法,不难得到:\[
    \bm{D}^{l} = \mathrm{diag}(d_{1}^{l},d_{2}^{l},\cdots,d_{n}^{l})
    \]
  \end{solution}

  %例2.2.16
  \begin{example}
设\(\bm{A} =
\begin{pmatrix}
  0 & 1 & 0 \\
  0 & 0 & 1 \\
  0 & 0 & 0 \\
\end{pmatrix}
\),$l$为正整数。计算\(\bm{J}^{l} \)
  \end{example}
  \begin{solution}
    通过直接计算知道:\[
 2
      \]
  \end{solution}
  \begin{remark}
    此例题告诉我们,如果有要求计算方幂的题目,可以试试是否在几次方幂后变为零矩阵。或者将原矩阵拆成两个矩阵的和,使其中一个矩阵在低次方幂后变为零矩阵。(此思路类似于微积分中用莱布尼茨公式求解高阶导数的方法。),参见~\ref{col:BinomialExpansionOfMatrix}。
  \end{remark}

  % 例2.2.17
  \begin{example}
    need to be completed soon 例2.2.17
  \end{example}
  \begin{solution}
    
  \end{solution}
  \begin{remark}
    一个秩为一的矩阵分解为行向量和列向量的乘积形式,从而可以利用乘法结合律简化计算。%此处运用到原书命题3.5.6,需要补全引用
  \end{remark}

  根据一般矩阵乘法的结合律,可以得到以下推论:
  \begin{corollary}{矩阵方幂的运算性质}
    \label{cor:2218}
    对任意的n阶方幂$\bm{A}$,以及任意的$k,l \in \mathrm{N}^{+}$,有:
    \begin{enumerate}
    \item \(\bm{A}^{k} \bm{A}^{l} = \bm{A}^{k+l} \)
    \item \( \left( \bm{A}^{k} \right)^{l} = \bm{A}^{kl}  \)
    \end{enumerate}
    
  \end{corollary}

  此外,若两个方阵符合可交换条件,则可以对它们和的方阵进行整数次的二项式展开:
  \begin{corollary}
    \label{col:BinomialExpansionOfMatrix}
    need to be completed
  \end{corollary}
  \begin{proof}
    need to be completed
  \end{proof}

  \subsection{转置}
\label{sec:224}

\begin{definition}{矩阵的转置}
  \label{def:TransposeOfMatrix}
  need to be completed
\end{definition}
\begin{note}
  从矩阵中每个数的角度,矩阵的转置似乎只是翻转了数字的排列。然而,若从行列向量的角度观察矩阵,
\end{note}



%%% Local Variables:
%%% mode: LaTeX
%%% TeX-master: "../main.tex"
%%% coding: utf-8
%%% End:
